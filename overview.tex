\section{Overview of the Problem}\label{mot}
\noindent
Given a set of real time tasks, each with a deadline, the problem is to schedule the tasks on the DRAM such that the fewest 
number of tasks miss their deadlines.

$i^{th}$ task instance in a real time system, denoted by $\tau_{i}$ can be represented with the help of the following 
parameters -
\[\tau_{i} = (A_{i}, E_{i}, D_{i}, P_{i})\]
where $A_{i}$ denotes the arrival time of $i^{th}$ task instance, $E_{i}$ denotes worst case execution time of $i^{th}$ task instance,
$D_{i}$ denotes deadline of $i^{th}$ task instance, $P_{i}$ denotes the time interval after which the next task instance 
arrives.

We present our problem with the help of a motivating example.
\subsection{Motivating Example}
\noindent
Let us consider a simple DRAM with two banks, B0 and B1, each bank having a set of four rows. 
The bank mapping of the two banks are as follows -
\begin{itemize}
 \item Bank B0 contains four sets of rows. Row 0 (addr 0-addr 3), Row 1 (addr 4-addr 7), Row 2 (addr 8-addr 11) and 
 Row 3 (addr 12-addr 15).
 \item Bank B1 contains four sets of rows. Row 0 (addr 16-addr 19), Row 1 (addr 20-addr 23), Row 2 (addr 24-addr 27) 
 and Row 3 (addr 28-addr 31).
\end{itemize}
We consider that our DRAM follows open page policy, i.e., memory requests which result in row hit will be allowed to access 
the DRAM. We consider two tasks T1 and T2 arriving at the same time instant and scheduled according to EDF at the processor.
Each task consists of some CPU and Memory instructions. In this example, we have denoted the CPU instructions with $<$C$>$ and 
the Memory instructions as $<$M addr$>$, which denotes a memory instruction at address ``addr''. We consider the task 
specifications given in Table \ref{tab1}. Assume each CPU instruction takes 100 cycles to execute,
each memory instruction takes 300 cycles to execute when there is a row miss and 200 cycles to execute when there is a row hit.
Since tasks are scheduled according to EDF, T2 will be allowed to access the CPU earlier. A detailed picture of the 
waiting queue at the processor, the processor, the memory buffer, the memory and the status of T1 and T2 at different
cycles is described in Table \ref{tab2}. It can be inferred from this example, that due to address mapping policy, all the 
memory requests are mapped to Bank B1. Again, due to the open row policy, some of T1's memory requests although arrived 
earlier, could not get access to memory. As a result, T1 cannot complete its execution within its deadline. 

If T1's memory request is allowed to execute in Bank B0, then T1 can easily complete its execution within its deadline.
But executing a memory request in some other memory bank incurs a lot of overheads. We propose a novel method of bank aware 
partitioning based on a cost function calculated locally taking into account all the overheads so that the tasks can complete 
execution within deadline. We discuss our proposal in the following subsection.

\begin{table}[t]
\caption{Task Specifications}
\begin{center}
\scalebox{0.9}{
 \begin{tabular}{|c|c|c|c|}\hline
 \bf{Task}& \bf{Instructions}& \bf{Bank}& \bf{Deadline} \\ 
 \bf{ID}  &             & \bf{Mapping}&      \\ \hline
 T1 & C & & 1800 cycles \\
    & M 25 & Bank1-Row2 & \\
    & M 20 & Bank1-Row1 & \\
    & C & & \\
    & M 30 & Bank1-Row3 & \\
    & M 28 & Bank1-Row3 & \\ \hline
 T2 & C & & 1500 cycles \\
    & M 30 & Bank1-Row3 & \\
    & M 29 & Bank1-Row3 & \\
    & C & & \\
    & M 24 & Bank1-Row2 & \\ \hline
 \end{tabular}
 }
 \end{center}
\label{tab1}
\end{table}

\begin{table}[t]
\centering
\caption{A snapshot of the entire system at different cycles}
\scalebox{0.9}{
\begin{tabular}{|c|c|c|c|c|c|}\hline
\bf{Cycle}&\bf{Waiting}&\bf{Processor}&\bf{Memory}&\bf{Memory}&\bf{Status} \\ 
        &\bf{Queue}&              &\bf{Buffer}&             &\\ \hline
       0& T1       & T2           &           &          &\\ \hline
      99& T1       & T2           &           &          &\\ \hline
     100&          & T1           &           &T2 $<$M 30$>$& T2-Row\\ 
        &          &              &           &Bank1 Row 3& Miss\\ \hline
     199&          & T1           &           &T2 $<$M 30$>$&\\ 
        &          &              &           &Bank1 Row3&\\ \hline
     200&          &              & T1        &T2 $<$M 30$>$&\\
        &          &              &           &Bank1 Row3&\\ \hline
     399&         &              & T1        &T2 $<$M 30$>$&\\ 
        &          &              &           &Bank1 Row3&\\ \hline
     400&          &              & T1        &T2 $<$M 29$>$& T2-Row\\
        &          &              &           &Bank1 Row3& Hit\\ \hline
     599&          &              & T1        &T2 $<$M 29$>$& \\
        &          &              &           &Bank1 Row3&\\ \hline
     600&          & T2           &           &T1 $<$M 25$>$& T1-Row\\ 
        &          &              &           &Bank1 Row2& Miss\\ \hline
     899&          &              & T2        &T1 $<$M 25$>$& \\
        &          &              &           &Bank1 Row2&\\ \hline
     900&          &              & T1        &T2 $<$M 24$>$& T2-Row\\ 
        &          &              &           &Bank1 Row2& Hit\\ \hline
    1099&          &              & T1        &T2 $<$M 24$>$& T2-complete \\ 
        &          &              &           &Bank1 Row2& execution\\ \hline
    1100&          &              &           &T1 $<$M 20$>$& T1-Row \\ 
        &          &              &           &Bank1 Row1& Miss\\ \hline
    1399&          &              &           &T1 $<$M 20$>$&\\
        &          &              &           &Bank1 Row1&\\ \hline
    1400&          & T1           &           &          &\\ \hline
    1499&          & T1           &           &          &\\ \hline
    1500&          &              &           &T1 $<$M 30$>$& T1-Row\\ 
        &          &              &           &Bank1 Row3& Miss\\ \hline
    1799&          &              &           &T1 $<$M 30$>$& \\
        &          &              &           &Bank1 Row3&\\ \hline
    1800&          &              &           &T1 $<$M 28$>$& T1-Deadline \\ 
        &          &              &           &Bank1 Row3& Miss\\ \hline     
\end{tabular}
}

\label{tab2}
\end{table}

\subsection{Detailed Methodology}
\noindent
We have implemented a two level scheduler, one at the processor and the other at the memory controller. We have applied EDF to 
schedule tasks at the processor, i.e., tasks whose deadline is earliest is placed at the front of the priority queue and are 
allowed to execute first. But as the memory requests arrive at the DRAM controller, they are not executed in the order 
they are scheduled. Open page policy prioritizes those memory requests which result in row buffer hit of tasks. We propose 
to keep a set of extra memory banks which will not participate in address mapping. These extra memory banks are also known as
the reserved banks. We formulate a cost function based on some
known task parameters and schedule tasks based on that cost function. Cost function, $Cost_{ij}$, is the cost to execute 
$i^{th}$ instruction of $j^{th}$ task. We define another variable, $C_{ijk}$, which is the remaining time after execution
of $i^{th}$ instruction of $j^{th}$ task in $k^{th}$ bank (where it is actually mapped) and is given by-
\[C_{ijk} = D_{j} - (t_{ik} + e_{iw} + e_{ik} + e_{rem})\]\label{eq1}
where $D_{j}$ : the deadline of $j^{th}$ task,\\
$t_{ik}$ : the time to copy the $i^{th}$ instruction in $k^{th}$ bank,\\
$e_{iw}$ : the maximum waiting time of $i^{th}$ task in the memory buffer due to presence of other tasks infront of the 
$i^{th}$ instruction in the memory buffer,\\
$e_{ik}$ : the execution time of $i^{th}$ instruction in $k^{th}$ bank if there is a row miss/hit,\\
$e_{rem}$ : the worst case execution time required by the remaining instructions of $j^{th}$ task.

Similarly, $C_{ijk'}$ is defined as the remaining time after execution of $i^{th}$ instruction of $j^{th}$ task in one of the 
extra memory banks or the reserved banks, say k'. Our aim is to execute the $i^{th}$ instruction of $j^{th}$ task instance in the bank that allows us to 
have greater remaining time so that there is a lesser chance of having a deadline miss. This is determined with the help of the 
cost function $C_{ij}$ which is given by-

\[Cost_{ij} = C_{ijk} - C_{ijk'}\]

If $Cost_{ij}$ is greater than 0, we allow the task to execute in its own bank, otherwise we schedule it to execute in any one 
of the reserved set of banks. Algorithm \ref{alg1} explains our proposed method.

\begin{algorithm}[t]
   \caption {\textcolor{black}{Scheduling Policy at the Memory Controller}}
   \label{alg1}
    \begin{algorithmic}[1]
      \Function{SchedMemReq}{$Task$ $t$, Instruction $I$}\Comment{$t$ - Task, $I$ - $i^{th}$ memory instruction of task $t$}
       \State {$t_{ik}$ - time to copy $i^{th}$ instruction in $k^{th}$ bank}
       \State {$e_{iw}$ = maximum waiting time of $t$ due to tasks which are in the waiting queue ahead of $t$}
       \State {$e_{ik}$ = execution time of $i^{th}$ instruction in $k^{th}$ bank in case of a row hit/miss}
       \State {$e_{ik'}$ = execution time of $i^{th}$ instruction in its own bank in case of a row hit/miss}
       \State {$e_{rem}$ = worst case execution time of all the remaining instructions}
       \State {$C_{itk}$ = $t.deadline$ - ($t_{ik}$ + $e_{iw}$ + $e_{ik}$ + $e_{rem}$)}\Comment{$C_{itk}$ denotes the total time
       remaining after executing $i^{th}$ instruction of task $t$ in $k^{th}$ bank}
       \State {$C_{itk'}$ = $t.deadline$ - ($e_{iw}$ + $e_{ik'}$ + $e_{rem}$)}\Comment{$C_{itk}$ denotes the total time
       remaining after executing $i^{th}$ instruction of task $t$ in its own bank}
       \State {$Cost_{ij}$ = $C_{ijk}$ - $C_{ijk'}$}\Comment{$Cost_{ijk}$ denotes the cost incurred in the process}
       \If {$Cost_{ij}$ $>$ 0}
           \State {Schedule $t$ in its own bank}
       \Else
           \State {Schedule $t$ in some reserved banks}
       \EndIf
      \EndFunction  
    \end{algorithmic}
\end{algorithm}

